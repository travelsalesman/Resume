%%%%%%%%%%%%%%%%%%%%%%%%%%%%%%%%%%%%%%%%%%%%%%%%%%%%%%%%%%%%%%%%%%%%%%%%
%%%%%%%%%%%%%%%%%%%%%% Simple LaTeX CV Template %%%%%%%%%%%%%%%%%%%%%%%%
%%%%%%%%%%%%%%%%%%%%%%%%%%%%%%%%%%%%%%%%%%%%%%%%%%%%%%%%%%%%%%%%%%%%%%%%

%%%%%%%%%%%%%%%%%%%%%%%%%%%%%%%%%%%%%%%%%%%%%%%%%%%%%%%%%%%%%%%%%%%%%%%%
%% NOTE: If you find that it says                                     %%
%%                                                                    %%
%%                           1 of ??                                  %%
%%                                                                    %%
%% at the bottom of your first page, this means that the AUX file     %%
%% was not available when you ran LaTeX on this source. Simply RERUN  %%
%% LaTeX to get the ``??'' replaced with the number of the last page  %%
%% of the document. The AUX file will be generated on the first run   %%
%% of LaTeX and used on the second run to fill in all of the          %%
%% references.                                                        %%
%%%%%%%%%%%%%%%%%%%%%%%%%%%%%%%%%%%%%%%%%%%%%%%%%%%%%%%%%%%%%%%%%%%%%%%%

%%%%%%%%%%%%%%%%%%%%%%%%%%%% Document Setup %%%%%%%%%%%%%%%%%%%%%%%%%%%%

% Don't like 10pt? Try 11pt or 12pt
\documentclass[10pt]{article}

% This is a helpful package that puts math inside length specifications
\usepackage{calc}

% Layout: Puts the section titles on left side of page
\reversemarginpar

%
%         PAPER SIZE, PAGE NUMBER, AND DOCUMENT LAYOUT NOTES:
%
% The next \usepackage line changes the layout for CV style section
% headings as marginal notes. It also sets up the paper size as either
% letter or A4. By default, letter was used. If A4 paper is desired,
% comment out the letterpaper lines and uncomment the a4paper lines.
%
% As you can see, the margin widths and section title widths can be
% easily adjusted.
%
% ALSO: Notice that the includefoot option can be commented OUT in order
% to put the PAGE NUMBER *IN* the bottom margin. This will make the
% effective text area larger.
%
% IF YOU WISH TO REMOVE THE ``of LASTPAGE'' next to each page number,
% see the note about the +LP and -LP lines below. Comment out the +LP
% and uncomment the -LP.
%
% IF YOU WISH TO REMOVE PAGE NUMBERS, be sure that the includefoot line
% is uncommented and ALSO uncomment the \pagestyle{empty} a few lines
% below.
%

%% Use these lines for letter-sized paper
\usepackage[paper=letterpaper,
            includefoot, % Uncomment to put page number above margin
            marginparwidth=1.2in,     % Length of section titles
            marginparsep=.05in,       % Space between titles and text
            margin=0.7in,               % 1 inch margins
            includemp]{geometry}

%% Use these lines for A4-sized paper
%\usepackage[paper=a4paper,
%            %includefoot, % Uncomment to put page number above margin
%            marginparwidth=30.5mm,    % Length of section titles
%            marginparsep=1.5mm,       % Space between titles and text
%            margin=25mm,              % 25mm margins
%            includemp]{geometry}

%% More layout: Get rid of indenting throughout entire document
\setlength{\parindent}{0in}

%% This gives us fun enumeration environments. compactenum will be nice.
\usepackage{paralist}

%% Reference the last page in the page number
%
% NOTE: comment the +LP line and uncomment the -LP line to have page
%       numbers without the ``of ##'' last page reference)
%
% NOTE: uncomment the \pagestyle{empty} line to get rid of all page
%       numbers (make sure includefoot is commented out above)
%
\usepackage{fancyhdr,lastpage}
\pagestyle{fancy}
\pagestyle{empty}      % Uncomment this to get rid of page numbers
\fancyhf{}\renewcommand{\headrulewidth}{0pt}
\fancyfootoffset{\marginparsep+\marginparwidth}
\newlength{\footpageshift}
\setlength{\footpageshift}
          {0.5\textwidth+0.5\marginparsep+0.5\marginparwidth-2in}
\lfoot{\hspace{\footpageshift}%
       \parbox{4in}{\, \hfill %
                    \arabic{page} of \protect\pageref*{LastPage} % +LP
%                    \arabic{page}                               % -LP
                    \hfill \,}}

% Finally, give us PDF bookmarks
\usepackage{color,hyperref}
\definecolor{darkblue}{rgb}{0.0,0.0,0.0}
\hypersetup{colorlinks,breaklinks,
            linkcolor=darkblue,urlcolor=darkblue,
            anchorcolor=darkblue,citecolor=darkblue}

%%%%%%%%%%%%%%%%%%%%%%%% End Document Setup %%%%%%%%%%%%%%%%%%%%%%%%%%%%


%%%%%%%%%%%%%%%%%%%%%%%%%%% Helper Commands %%%%%%%%%%%%%%%%%%%%%%%%%%%%

% The title (name) with a horizontal rule under it
%
% Usage: \makeheading{name}
%
% Place at top of document. It should be the first thing.
\newcommand{\makeheading}[1]%
        {\hspace*{-\marginparsep minus \marginparwidth}%
         \begin{minipage}[t]{\textwidth+\marginparwidth+\marginparsep}%
                {\Large \bfseries #1}\\[-0.15\baselineskip]%
                 \rule{\columnwidth}{1pt}%
         \end{minipage}}

% The section headings
%
% Usage: \section{section name}
%
% Follow this section IMMEDIATELY with the first line of the section
% text. Do not put whitespace in between. That is, do this:
%
%       \section{My Information}
%       Here is my information.
%
% and NOT this:
%
%       \section{My Information}
%
%       Here is my information.
%
% Otherwise the top of the section header will not line up with the top
% of the section. Of course, using a single comment character (%) on
% empty lines allows for the function of the first example with the
% readability of the second example.
\renewcommand{\section}[2]%
        {\pagebreak[2]\vspace{1.3\baselineskip}%
         \phantomsection\addcontentsline{toc}{section}{#1}%
         \hspace{0in}%
         \marginpar{
         \raggedright \scshape #1}#2}

% An itemize-style list with lots of space between items
\newenvironment{outerlist}[1][\enskip\textbullet]%
        {\begin{enumerate}[#1]}{\end{enumerate}%
         \vspace{-.6\baselineskip}}

% An environment IDENTICAL to outerlist that has better pre-list spacing
% when used as the first thing in a \section
\newenvironment{lonelist}[1][\enskip\textbullet]%
        {\vspace{-\baselineskip}\begin{list}{#1}{%
        \setlength{\partopsep}{0pt}%
        \setlength{\topsep}{0pt}}}
        {\end{list}\vspace{-.6\baselineskip}}

% An itemize-style list with little space between items
\newenvironment{innerlist}[1][\enskip\textbullet]%
        {\begin{compactenum}[#1]}{\end{compactenum}}

% To add some paragraph space between lines.
% This also tells LaTeX to preferably break a page on one of these gaps
% if there is a needed pagebreak nearby.
\newcommand{\blankline}{\quad\pagebreak[2]}

%%%%%%%%%%%%%%%%%%%%%%%% End Helper Commands %%%%%%%%%%%%%%%%%%%%%%%%%%%

%%%%%%%%%%%%%%%%%%%%%%%%% Begin CV Document %%%%%%%%%%%%%%%%%%%%%%%%%%%%

\begin{document}
\makeheading{Rajiv Kadaba}

\section{Contact Information}
%
% NOTE: Mind where the & separators and \\ breaks are in the following
%       table.
%
% ALSO: \rcollength is the width of the right column of the table
%       (adjust it to your liking; default is 1.85in).
%
\newlength{\rcollength}\setlength{\rcollength}{2.15in}%
%
\begin{tabular}[t]{@{}p{\textwidth-\rcollength}p{\rcollength}}
229 Easy St. Apt. B  & \textit{Voice:} (512) 573-0179 \\
Mountain View, CA 94043 USA       & \textit{E-mail:} \href{mailto:rajiv.kadaba@gmail.com}{rajiv.kadaba@gmail.com}\\
& \textit{Web:} \href{http://users.ece.utexas.edu/~rkadaba}{users.ece.utexas.edu/$\sim$rkadaba}\\
\end{tabular}

\section{Education}
%
\textbf{Master of Science, \href{http://www.ece.utexas.edu/}{Electrical and Computer Engineering}, December 2009}

        \href{http://www.utexas.edu/}{The University of Texas at Austin}

\blankline

\textbf{Bachelor of Engineering, Electronics, May 2007}

    \href{http://www.vjti.ac.in/}{Veermata Jijabai Technological Institute (VJTI)}

\section{Courses}
Convex Optimization, Data Mining,  Computer Security and Privacy, Formal Methods in Distributed Systems, Engineering Programming Languages, Computer Architecture, VLSI, Robotics

\section{Professional Experience}
%
\textbf{\href{http://www.micrsoft.com/}{Microsoft Corporation}}
\begin{outerlist}
\item[] \textit{Software Development Engineer in Test, \href{http://www.microsoftmediaroom.com/}{Mediaroom}} \hfill \textbf{August 2009 to Present}
    \begin{innerlist}
    \item Architected a test automation framework for multi platform integration and performance testing of a heterogenous family of IPTV clients.
    \item Developed a modeling system of the mediaroom AV pipeline to predict user experience issues and increase testability.
    \item Built data visualization tools to help diagnose A/V quality.
    \end{innerlist}
\end{outerlist}
\begin{outerlist}
\item[] \textit{Software Development Engineer in Test Intern, \href{http://www.microsoftmediaroom.com/}{Mediaroom}} \hfill \textbf{June 2008 to August 2008}
    \begin{innerlist}
    \item Designed and implemented HDMI test framework which is flexible, portable and scalable along with an UI. 
    \item Developed HD frame capture application by porting Directshow Interfaces to .NET.
    \item Worked as part of team developing test infrastructure for Mircosoft's IPTV platform.
    \end{innerlist}
\end{outerlist}



\section{Academic Experience}
\textbf{\href{http://www.utexas.edu}{The University of Texas at Austin}}
\begin{outerlist}

\item[] \textit{Research Assistant, \href{http://www.lips.utexas.edu/}{The Laboratory for Intelligent Processes and Systems}}
     \hfill \textbf{Summer 2009}
\begin{innerlist}
\item Conducted research in the field of multi agent systems and the semantic web, specifically identity and trust in social networks.
\item Developed software in python to build data sets, prototype algorithms and visualize results.
\item Presented laboratory research at conferences and assisted with writing of grant proposals.
\end{innerlist}

\item[] \textit{Teaching Assistant, \href{http://www.ece.utexas.edu/}{Electrical and Computer Engineering}}
     \hfill \textbf{Spring 2008 to Spring 2009}
\begin{innerlist}
\item Responsible for holding office hours, delivering lectures, providing model solutions to assignments, designing labs and grading (EE360C Algorithms) for 43 students.
\item Responsible for grading software requirements documents and facilitating in class exercises for undergraduate and graduate students (EE382C/EE361Q Requirements Engineering).
\item Independently, planned and developed a supplementary Instruction course in Electrostatics, Electromagnetism and Optics (GE208L).Instructed and managed a class of 22 students for a period of one semester.
\item Received a student evaluation of 4.4/5.0.
\end{innerlist}
\end{outerlist}

\blankline

\textbf{\href{http://www.barc.ernet.in/}{Bhabha Atomic Research Centre}}
\begin{outerlist}
\item[] \textit{Research Intern, Division of Remote Handling and Robotics} \hfill \textbf{Spring 2007}
\begin{innerlist}
\item Developed model of small payload serial link flexible manipulator and characterized its
oscillation dynamics.
\item Simulated control schemes in Matlab as proof of concept for a prototype.
\item Advisor: Dr. Debanik Roy
\end{innerlist}
\end{outerlist}


\section{Publications}
%
R. Kadaba, S Budalakoti, D. DeAngelis, and K. S. Barber. Modeling Virtual Footprints. In the Proceedings of The Workshop on Trust in Agent Societies at The Ninth
International Conference on Autonomous Agents and Multiagent Systems (AAMAS-
2010); Toronto, Canada; May 10-14, 2010.

\section{Projects}
%
Multivariate Statistical Approach to Reservoir Classification \hfill \textbf{Spring 2008}
\begin{outerlist}
\item Investigated the use of model trees and gradient boosting in predicting amount of gas recoverable from reservoirs in
the GASIS data set.
\item Achieved a prediction accuracy of over 99%
\end{outerlist}

\blankline

Early Design Planning and Circuit Feasibility Analysis of OpenSPARC T1 Core \hfill \textbf{Spring 2008}
\begin{outerlist}
\item Developed detailed floorplan of the Stream Processing Unit cluster.
\item Performed Power Estimation and Critical Path analysis on assigned cluster for Subthreshold
operation.
\end{outerlist}

\blankline

Design and Verification of a Reconfigurable Bloom Filter IP Core \hfill \textbf{Fall 2007}
\begin{outerlist}
\item Implemented fast pattern matching engine for Deep Packet Inspection using Bloom Filters
which are always optimal in verilog.
\item Designed Multi banked, pipelined SRAM architecture to allow for a throughput of 500MB/s.
\item Validated against software model written in C++.
\end{outerlist}

\blankline

Human Machine Interface with Inertial Sensors using Optimal Estimation
Techniques \hfill \textbf{Fall 2006}
\begin{outerlist}
\item Explored the advantages of inertial sensors in gesture recognition applications.
\item Designed and built framework to allow humans to interact with existing software intuitively.
\end{outerlist}

\section{Skills}
%
Programming Languages: Assembly (x86, MSC51, AVR), C, C++, C\# .NET 4.0, Python

\blankline

Platforms: Windows, Linux (Fedora)

\blankline

Tools: Matlab/Simulink, Weka, MSC ADAMS

\blankline

Web Development: Silverlight/WPF, PHP, MySQL, MS SQL

\blankline

Integrated Circuit CAD: Cadence Virtuoso, Synopsys VCS, HSPICE

\section{Achievements}
%
First Place Winner, Robot Localization, IRocha, IIIT Hyderabad, 2007

\blankline

Best Design Winner, IClean, Techfest , IIT Bombay, 2007

\blankline

Second Place Winner, Level 3 Robotics, IEEE ISAC, 2005

\blankline

Second Place Winner, Level 3 Robotics, IEEE Brainwaves, 2004

\section{Professional Memberships \& Service}
%
Institute for Electrical and Electronic Engineers (IEEE) Student Member 2006 - Present

\blankline

Technical Head, IEEE VJTI Student Branch, 2006 - 2007 	

\end{document}

%%%%%%%%%%%%%%%%%%%%%%%%%% End CV Document %%%%%%%%%%%%%%%%%%%%%%%%%%%%%
